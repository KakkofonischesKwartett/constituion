\documentclass[ngerman, fontsize=12pt, parskip=half, footsepline]{scrartcl}
%\usepackage{ngerman}
\usepackage[automark]{scrpage2}
\usepackage[german]{babel}
\usepackage{graphicx}
\usepackage{geometry}
\usepackage[utf8]{inputenc}
\usepackage[T1]{fontenc}
\usepackage{lmodern}
\usepackage{amsmath, amsthm, amssymb}

\geometry{a4paper,left=35mm,right=20mm, top=3cm, bottom=3cm}
\linespread{1.25}
\setkomafont{section}{\fontsize{18pt}{21pt}\selectfont}
\setkomafont{subsection}{\fontsize{13pt}{18pt}\mdseries\bfseries}
\setkomafont{subsubsection}{\normalsize\mdseries\bfseries}
\setkomafont{paragraph}{\normalsize\mdseries}
\setkomafont{subparagraph}{\normalsize\mdseries}

\renewcommand{\thesection}{\Roman{section}}

\clearscrheadfoot
\pagestyle{scrplain}

\title{Verfassung des Kakkofonischen Staates}
\author{Das Kakkofonische Kwartett}
\date{\today}

\begin{document}
\maketitle{}
\section*{Präambel}

Im Bewußtsein seiner Verantwortung, von Bier beseelt, als alleinberechtigtes Glied, Kraft seiner
unendlichen Weisheit hat das Kakkofonische Kwartett sich dieses Gerät ausgedacht. Weil das alte weg ist.\\
Dieses Gerät gilt für alle Kakkofonier – ihr Fotzen.

\section{Grundrechte, primär Grundpflichten}

\subsection*{Artikel 1}
Die Würde des Stuhlgangs ist unantastbar. Die des gemeinen oder auch fiesen Kakkofoniers auch nicht.

\subsection*{Artikel 2}
Ehre deinen Stuhl! Dem Stuhl eines jeden Kakkofoniers, ganz besonders dem des Königs und der Kwarttetanten, werden magische Kräfte zugeschrieben. Ebendso den Flatulenzen, besonders laute oder geruchsvolle Darmwinde werden von allen anwesenden Kakkofoniern mit einem respektvollen „JO!“ und eventuell mit einer Kakkofonischen Fanfare gewürdigt.

\subsection*{Artikel 3}
Die Würde des Bieres ist unantastbar, es zu verschütten wird mit  Entzug des Kakkofoniertums bestraft. Bier zu trinken ist die heiligste aller kakkofonischen Pflichten, ihr nicht nach zu kommen ist dumm und wird ebendso mit Entzug des Kakkofoniertums bestraft.

\subsection*{Artikel 4}
Die Ur-Kakkofonier, jedes Mitglied des Kakkofonischen Kwartetts sowie Träger eines sonstigen Kakkofonischen Amtes sind automatisch Kakkofonier.

\subsection*{Artikel 5}
Jeder Kakkofonier ist vor dem Geschwätz gleich. Nur der Kakkofonische König ist erheblich gleicher.

\subsection*{Artikel 6}
Jeder Kakkofonier hat ein Grundrecht auf exzessiven Genuss von Bier, Koffein, Nikotin und Klopapier.

\section{Kakkofonien}

\subsection*{Artikel 1: Staatsgebilde}
Kakkofonien ist ein temporärer, nicht-territorialer Staat. Die Staatsform ist eine absolutistische Wahlmonarchie. Völlig oligarchisch und unter Ausschluss der unteren, untersten und allerunteresten Schichten sowie bewegt-bildvebreitender und -verengender  Medien.

\subsection*{Artikel 2: Ehrfürchtigkeit}
Der Kakkofonische Staat ist ein erfüchtiger Gottesstaat. Die Gottheiten des Kakkofonischen Staates sind:
\begin{itemize}
	\item Der heilige Stuhl
	\item Toiletten im Allgemeinen
	\item Das Bier
	\item Der Pegelmolch
	\item Der ISWI Infopoint seit anno latrinum 2 nicht mehr
\end{itemize}
\section{Organe und Innereien}

\subsection*{Artikel 1: Das Kakkofonische Kwartett}
Das Kakkofonische Kwarttet (DKK) setzt sich aus vier Kakkofoniern zusammen. Feste Mitglieder sind die drei Ur-Kakkofonier, die im Jahr der Kakkofonischen Entstehung, anno latrinum, aus einem Tiefspüler gekrochen sind. Die drei Ur-Kakkofonier sind Lukas, Max und Bo. Das vierte Mitglied des Kwartetts ist ein völlig beliebiger, dahergelaufener Kakkofonier der von den drei Ur-Kakkofoniern mit einer drei-drittel Mehrheit gewählt wird. Die primäre Pflicht des DKKs besteht darin möglichst viele Abstimmungen, anhand obskurer Abstimmungsmoditäten (siehe Absatz IV: Abstimmungsmoditäten) erfolgreich durchzuführen und die Verfassung zu verbreiten. Eine weitere Aufgabe besteht darin durch exzessive Gehirnwäsche, Einschüchterung oder Bestechung möglichst viele Personen und ISTUFFler zur kakkofonischen Weltanschauung zu bekehren und sie damit zu völlig willens- und rechtelosen Untertanen zu machen.

Eine Expansion des DKKs ist prinzipiell durch eine Sonstige Abstimmung des DKKs möglich, allerdings würde eine solche Entscheidung die Machtbasis der Ur-Kakkofonier mindern. Aber wer will das denn?

Alle Mitglieder des DKKs sind durch sonstige Abstimmung wahlweise gegen sämtliche Kakkofonische Gesetze immun.

\subsection*{Artikel 2: Der Kakkofonsiche König}
Der Kakkofonische König (DKK) wird vom das DKK gewählt. Er ist der mehr oder weniger alleiniger Herrscher über alle Kakkofonier.

Der Stuhl des Königs, sowie alle sonstigen Feststoffe die den Körper des Königs unterhalb seines Bauchnabels verlassen, werden als heiliger Stuhl bezeichnet.
Die Legislaturperiode eines der DKKs  ist gleich dem Quotienten der zum Zeitpunkt der ersten Königswahl einer Nacht verbleibenden Sendezeit und der Anzahl der Kwarttetsmitglieder.

Die Rechte des der DKKs:
\begin{enumerate}
	\item Der DKK ist befugt während seiner Amtszeit alleine über die Musikauswahl zu verfügen. Es 	ist ihm theoretisch gestattet diese Aufgabe zu delegieren. Ein solches Vorgehen wäre aber dumm und würde mit dem absoluten Respektsverlust des das DKKs geahndet.

	\item Um das Gesprächsniveau auf einem angemessenen Niveau zu halten. Hat der DKK jederzeit das Recht Gesprächsregeln einzuführen. Verstöße gegen solche Regeln werden je nach Lust und Laune durch das das DKK bestraft. Über das Strafmaß, oder im Bierfall die Strafmaß entscheidet das das DKK durch eine sonstige Abstimmung.
\end{enumerate}

Der der DKK ist zu Allen Zeiten mit „Eure Königlichkeit“ anzureden. Eine Zuwiderhandlung wird nicht bestraft.

Die Insignien der Macht des der DKKs sind das Elfenbeinerne Zepter der Kakkofonischen Allmacht sowie seine höchstpersönliche Krone, die auf seinem, zu Recht, alleröchst erhobenem Haupthaupt draufsitzt.


\subsection*{Artikel 3: Sonstige Ämter}

Sonstige Kakkofonische Ämter sind in ihren Rechten und Pflichten nicht weiter geregelt, diese sind bei Vergabe des Amtes durch das DKK zur erfinden oder sich auszudenken. Mitgliedern des das DKKs ist es nicht gestattet, Sonstige Ämter auszuüben.
Amtsinhaber, deren Amtsbezeichnung mit einem „K“ beginnt, ist bei Einführung mit einer Kakkofonischen Fanfare besonderer Respekt zu erweisen.
Sonstige Ämter werden durch das das DKK durch eine Sonstige Abstimmung vergeben. Die Ablehnung eines Sonstigen Amtes ist nicht gestattet und wäre mit einer Beleidigung des Heiligen Stuhls gleichzusetzen.



\subsection*{Artikel 4: Normalo Kakkofonier}
Normalo Kakkofonier ist jeder, der zu irgendeinem Zeitpunkt in seinem nutzlosen Leben den Satz „Ich will Kakkofonier sein!“ gerufen hat. Oder durch Geburt. Ist einem Normalo Kakkofonier das Kakkofoniertum aberkannt worden, kann es ihm nur durch das das DKK wieder zurück gegeben werden. Ätsch!


Die Hauptaufgabe des Kakkofonischen Kwarttets ist es in Abstimmung abzustimmen. Wann, und nach welchen Modalitäten es zu Abstimmungen kommt wird im Abschnitt Abstimmungsmoditäten, erläutert.

\section{Abstimmungsmoditäten}

Grundsätzlich gilt:
Für alle Abstimmungen sind nur Mitglieder des das DKKs abstimmungsberechtigt.
Für alle Abstimmungen ist das Prinzip des Rekursiven Suffragiums anzuwenden.
Alle Abstimmungen sind unfrei und ungeheim. Sie werden ausschließlich durch das das DKK kontroliert und gegebenenfalls korrigiert. Ätsch!

\subsection*{Artikel 1: Königswahl}
Prinzipiell kann jeder Kakkofonier zum König gewählt werden. Ein nicht-Mitglied des das DKK zu wählen wäre aber dumm und würde von keinem noch so betrunkenen Mitglied des das DKKs auch nur ansatzweise in Erwägung gezogen werden.\\
Die Wahl des Königs erfolgt durch lautes Ausrufen des Vor- und/oder Künstlernamens des favorisierten Kandidaten. Somit ist es die einzige Abstimmung im Kakkofonischen Staat, bei der nicht mit „dafür“ oder „dagegen“ abgestimmt werden kann.\\
Es nicht möglich ein- und denselben König innerhalb eines Jahre öfter als einmal zu wählen.
Ist ein König gewählt, wird er durch lautes Aufsagen der feierlichen Formel „Möge sein Fehlklang auf ewig in den unendlichen Phasenfallen des hsf widerhallen.“ und einem frei erfundenen Schwur auf den Heiligen Stuhl in sein Amt eingeführt.

\subsection*{Artikel 2: Antraxabstimmungen}
Prinzipiell ist jedes Mitglied des das DKK befähigt, sofern physisch dazu in der Lage, einen Antrag vorzubringen, der der DKK ist deutlich befähigter.\\
Über Anträge zu Anträgen ist nach der Antraxabstimmungsordnung abzustimmen.
Im Zweifelsfall des Zweifels, oder eine unordnungsgemäßen Uneinigkeit hat der amtierende der DKK zwei Stimmen.

Es wird mit einfacher Mehrheit entschieden.

\subsection*{Artikel 3: Sonstige Abstimmungen}
Es gelten die Grundsätzlichen Abstimmungsmoditäten.\\

\subsection*{Artikel 4: Verfassungsveränderungen oder Andersmachungen}
Um die Verfassung zu ändern oder anders zu machen muss ein Antrag von einem Inhaber eines sonstigen Amtes vor das  das DKK vorgebracht werden.
Der Vorbringer des Antrages muss zuvor vom das DKK durch Bestechung, Einschüchterung oder physische Gewalt zur Vorbringung des Antrages genötigt worden sein.
Das das DKK wird dann der Regel des Rekursiven Suffragiums und der allgemeinen Abstimmungordnung über den Antrag abstimmungen.\\

\vspace{1cm}
\begin{tabular}{lp{2em}l}
 \hspace{5cm}   && \hspace{4cm} \\\cline{1-1}\cline{3-3}
 Ort, Datum     && Ur-Kakfonier Lukas
\end{tabular}\\

\vspace{1.5cm}
\begin{tabular}{lp{2em}l}
 \hspace{5cm}   && \hspace{4cm} \\\cline{1-1}\cline{3-3}
 Ort, Datum     && Ur-Kakfonier Max
\end{tabular}

\vspace{1.5cm}
\begin{tabular}{lp{2em}l}
 \hspace{5cm}   && \hspace{4cm} \\\cline{1-1}\cline{3-3}
 Ort, Datum     && Ur-Kakfonier Bo
\end{tabular}

\end{document}
